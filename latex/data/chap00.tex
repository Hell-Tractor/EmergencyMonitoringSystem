\chapter{任务概述}

\section{目标}

本系统开发旨在构建一套面向视觉感知与智能图像处理的高可靠性嵌入式监控系统,以实现对危险情况的实时、精确监控、智能识别与高效信息传输。

在图像处理与感知系统方面,实现通过摄像头采集图像数据,并支持一定程度上的变焦功能,能够根据指令自动变焦。利用深度学习技术对接收到的图像进行危险情况检测。

在数据传输与显示系统方面,通过多个WiFi中继器构建环形拓扑,以扩大无线网络覆盖范围并增强网络可靠性,确保单个中继器崩溃时不影响系统正常运行。最后通过网页将内容显示在终端设备上。

\section{需求概述}

本项目的总需求为开发一个能够实现危险情况实时监控与智能识别的嵌入式系统。该系统需支持图像的高效处理、数据的可靠传输以及信息的直观显示,以提供全面的危险情况预警与管理能力。

功能需求方面,系统需支持图像处理模块的整个业务流程操作,例如图像的采集、危险识别、变焦控制与区域裁剪;同时,还需要支持数据传输模块的通信链路管理与数据转发业务流程操作;以及显示模块的信息接收与呈现业务流程操作。具体而言,系统需支持对潜在危险的实时检测、在检测到危险时自动进行变焦以捕获清晰图像、在变焦极限时进行区域裁剪、并确保信息通过无线网络可靠传输至显示设备进行呈现。

性能需求方面,图像处理模块在处理高峰期需能满足多路视频流的并发处理需求,确保危险情况的实时识别与快速响应。数据传输模块则要求在野外远距离环境下能维持高带宽与低延迟的数据传输,保障信息的及时性。

安全性需求方面,系统要求保护图像数据与处理结果的完整性与隐私性,防止未经授权的访问与篡改。软件应具备安全性检查机制,对于外部接入的摄像头指令或显示请求,需进行严格验证;对于非法访问内部处理服务器的请求,应进行有效拦截。同时,系统需具备对处理负荷的动态负载均衡与限制能力,以防止对服务器性能的冲击攻击,确保系统在持续运行中的稳定可靠。